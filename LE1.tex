\documentclass[../Notas.tex]{subfiles}
\graphicspath{{\subfix{../images/}}}

\begin{document}

\subsection{Exercícios}
\begin{enumerate}
    \item Seja $(\Omega, \mathcal{A}, P)$ um espaço de probabilidade, onde $\mathcal{A}$ é a $\sigma$-álgebra de todos os subconjuntos de $\Omega$ e $P$ é uma medida de probabilidade que associa a probabilidade $p>0$ a cada conjunto de um ponto de $\Omega$.
    \begin{enumerate}[a)]
        \item Mostre que $\Omega$ deve ter um número finito de pontos [\textit{Sugestão:} mostre que $\Omega$ não pode ter mais de $p^{-1}$ pontos.]
        \item Mostre que se $n$ é o número de pontos em $\Omega$, então $p$ deve ser igual a $n^{-1}$.
    \end{enumerate}
    \item Pode-se construir um modelo para um \textit{spinner} aleatório tomando um espaço uniforme de probabilidade sobre a circunferência de um círculo de raio 1, de modo que a probabilidade de que o ponteiro do \textit{spinner} pare sobre um arco de comprimento $s$ é $s/2\pi$. Suponha que o círculo esteja dividido em 37 zonas numeradas de 1 a 37. Determine a probabilidade de que o ponteiro pare sobre uma zona de número par.
    \item Considere um ponto escolhido ao acaso sobre um quadrado unitário. Determine a probabilidade de que o ponto esteja no triângulo limitado por $x=0, y=0$ e $x+y=1$.
    \item Seja $P$ um ponto escolhido ao acaso sobre um círculo unitário. Determine a probabilidade de que $P$ esteja no setor angular de 0 a $\pi/4$ radianos.
    \item Uma caixa contém 10 bolas numeradas de 1 a 10. Extrai-se ao acaso uma bola da caixa. Determine a probabilidade de que o número da bola seja 3, 4 ou 5.
    \item Suponha que se lance um par de dados e que os 36 resultados possíveis são igualmente prováveis. Determine a probabilidade de que a soma dos números observados seja par.
    \item Suponha que $A$ e $B$ sejam eventos tais que $P(A) = 2/5$, $P(B) = 2/5$ e $P(A \cup B) = 1/2$. Determine $P(A \cap B)$.
    \item Se $P(A) = 1/3$, $P(A \cup B) = 1/2$ e $P(A \cap B) = 1/4$, determine $P(B)$.
    \item Suponha que se escolha ao acaso um ponto sobre um quadrado unitário. Seja $A$ o evento de que o ponto está no triângulo limitado por $y = 0$, $x = 1$ e $x = y$, e $B$ o evento de que o ponto está no retângulo com vértices em $(0,0)$, $(1,0)$, $(1, 1/2)$ e $(0,1/2)$. Determine $P(A \cup B)$ e $P(A \cap B)$.
    \item Suponha que temos quatro cofres, cada um com 2 gavetas. Os cofres 1 e 2, têm uma moeda de ouro em uma gaveta e uma de prata na outra. O cofre 3, tem duas moedas de ouro e, o cofre 4 tem duas de prata. Escolhe-se um cofre ao acaso, abre-se uma gaveta e encontra-se uma moeda de ouro. Determine a probabilidade de que a outra gaveta contenha: 
    \begin{enumerate}[a)]
    \item Uma moeda de prata;
    \item Uma moeda de ouro.
    \end{enumerate}
    \item Uma caixa contém 10 bolas das quais 6 são pretas e 4 são brancas. Remove-se três bolas sem observar suas cores. Determine a probabilidade de que uma quarta bola removida da caixa seja branca. Assuma que as 10 bolas são igualmente prováveis de serem removidas da caixa.
    \item Para uma caixa de mesma composição que a do exercício 11, determine a probabilidade de que todas as 3 primeiras bolas removidas sejam pretas, sabendo-se que pelo menos uma delas é preta.
    \item Suponha que uma fábrica tem duas máquinas $A$ e $B$, responsáveis, respectivamente, por 60\% e 40\% da produção total. A máquina $A$ produz 3\% de itens defeituosos, enquanto a máquina $B$ produz 5\% de itens defeituosos. Determine a probabilidade de que um dado item defeituoso foi produzido pela máquina $B$.
    \item Um estudante se submete a um exame de múltipla escolha no qual cada questão tem 5 respostas possíveis das quais exatamente uma é correta. O estudante seleciona a resposta correta se ele sabe a resposta. Caso contrário, ele seleciona ao acaso uma resposta entre as 5 possíveis. Suponha que o estudante saiba a resposta de 70\% das questões.
    \begin{enumerate}[a)]
    \item Qual a probabilidade de que o estudante escolha a resposta correta para uma dada questão?
    \item Se o estudante escolhe a resposta correta para uma dada questão, qual a probabilidade de que ele sabia a resposta?
    \end{enumerate}
    \item Suponha que se escolha ao acaso um ponto sobre um quadrado unitário. Sabendo-se que o ponto está no retângulo limitado por $y = 0$, $y = 1$ e $x = 0$ e $x = 1/2$, qual é a probabilidade de que o ponto esteja no triângulo limitado por $y = 0$, $x = 1/2$ e $x + y = 1$?
    \item Suponha que uma caixa contenha $r$ bolas vermelhas e $b$ bolas pretas. Extrai-se ao acaso uma bola da caixa e a seguir extrai-se, também ao acaso, uma segunda bola dentre as que ficaram na caixa. Determine a probabilidade de que:
    \begin{enumerate}[a)]
    \item Ambas as bolas sejam vermelhas;
    \item A primeira bola seja vermelha e a segunda preta;
    \item A primeira bola seja preta e a segunda vermelha;
    \item Ambas as bolas sejam pretas.
    \end{enumerate}
    \item Uma caixa contém 10 bolas vermelhas e 5 pretas. Extrai-se uma bola da caixa. Se ela é vermelha, ela é recolocada na caixa. Se é preta, além de recolocá-la na caixa, adiciona-se duas bolas pretas à caixa. Determine a probabilidade de que uma segunda bola extraída da caixa seja
    \begin{enumerate}[a)]
    \item vermelha;
    \item preta.
    \end{enumerate}
    \item Extrai-se duas bolas, com reposição da primeira, de uma caixa contendo 3 bolas brancas e 2 bolas pretas.
    \begin{enumerate}[a)]
    \item Construa um espaço amostral com pontos igualmente prováveis para este experimento
    \item Determine a probabilidade de que as bolas extraídas sejam da mesma cor.
    \item Determine a probabilidade de que, pelo menos, uma das bolas extraídas seja branca
    \end{enumerate}
    \item Resolva o exercício 18, caso a primeira bola não seja reposta na caixa.
    \item Uma caixa contém 3 bolas brancas e 2 bolas pretas. Selecionam-se duas bolas sem reposição.
    \begin{enumerate}[a)]
    \item Calcule a probabilidade de que a segunda bola seja preta, dado que a primeira é preta.
    \item Calcule a probabilidade de que a segunda bola seja da mesma cor da primeira.
    \item Calcule a probabilidade de que a primeira bola seja branca, dado que a segunda é branca.
    \end{enumerate}
    \item Suponha que existisse um teste para câncer com probabilidade de que 90\% das pessoas com câncer e 5\% das pessoas sem câncer reagem positivamente. Admita que 1\% dos pacientes de um hospital têm câncer. Qual a probabilidade de que um paciente escolhido ao acaso, que reage positivamente a esse teste, realmente tenha câncer?
    \item Suponha que se lança três moedas idênticas e perfeitamente equilibradas. Seja $A_i$ o evento de observar cara na $i$-ésima moeda. Mostre que os eventos $A_1$ , $A_2$ e $A_3$ são mutuamente independentes.
    \item Suponha que as seis faces de um dado têm igual probabilidade de ocorrência e que sucessivos lançamentos do dado são independentes. Construa um espaço de probabilidade para o experimento composto de três lançamentos do dado.
    \item Suponha que $A$, $B$ e $C$ são eventos mutuamente independentes e $P(A \cap B) \neq 0$. Mostre que $P(C|A \cap B) = P(C)$.
    \item Experiência mostra que 20\% das pessoas que fazem reservas de mesa num certo restaurante deixam de comparecer. Se o restaurante tem 50 mesas e aceita 52 reservas, qual a probabilidade de que seja capaz de acomodar todos os fregueses?
    \item Um certo componente de um motor de foguete falha 5\% das vezes quando o motor é acionado. Para obter maior confiabilidade no funcionamento do motor, usa-se $n$ componentes em paralelo, de maneira que o motor falha somente se todos os componentes falharem. Suponha que as falhas dos componentes sejam independentes uma das outras. Qual é o menor valor de n que pode ser usado para garantir que o motor funcione 99\% das vezes?
    \item Existem 4 reis num baralho de 52 cartas. Extrai-se uma carta do baralho, registra-se o seu valor e a seguir repõe-se a carta extraída. Este procedimento é repetido 4 vezes. Determine a probabilidade de que existam exatamente 2 reis entre as cartas selecionadas, sabendo-se que entre elas existe pelo menos um rei.
    \item Mostre que se $A$, $B$ e $C$ são eventos tais que $P(A \cap B \cap C) \neq 0$ e $P(C|A \cap B)$ = $P(C|B)$, então $P(A|B \cap C) = P(A|B)$.
    \item Um homem dispara 12 tiros independentes num alvo. Qual a probabilidade de que ele atinja o alvo pelo menos uma vez, se tem probabilidade $9/10$ de atingir o alvo em qualquer tiro?
    \item No Exercício 29 qual a probabilidade de que o alvo seja atingido pelo menos duas vezes, sabendo-se que o mesmo foi atingido pelo menos uma vez?
    \end{enumerate}

\end{document}