\documentclass[../Notas.tex]{subfiles}
\graphicspath{{\subfix{../images/}}}

\begin{document}

\subsection{Exercícios}
\begin{enumerate}
    \item Seja $(\Omega, \mathcal{A}, P)$ um espaço de probabilidade, onde $\mathcal{A}$ é a $\sigma$-álgebra de todos os subconjuntos de $\Omega$ e $P$ é uma medida de probabilidade que associa a probabilidade $p>0$ a cada conjunto de um ponto de $\Omega$.
    \begin{enumerate}[a)]
        \item Mostre que $\Omega$ deve ter um número finito de pontos [\textit{Sugestão:} mostre que $\Omega$ não pode ter mais de $p^{-1}$ pontos.]
        \item Mostre que se $n$ é o número de pontos em $\Omega$, então $p$ deve ser igual a $n^{-1}$.
    \end{enumerate}
    \item Pode-se construir um modelo para um ``spinner'' aleatório
\end{enumerate}

\end{document}