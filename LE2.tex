\documentclass[../Notas.tex]{subfiles}
\graphicspath{{\subfix{../images/}}}

\begin{document}

\subsection{Exercícios - combinatória}

\begin{enumerate}
    \item O código Morse consiste de uma sequência de pontos e traços em que repetições são permitidas.
    \begin{enumerate}[a)]
    \item Quantas letras podem ser codificadas usando exatamente $n$ símbolos?
    \item Qual é o número de letras que se pode codificar usando $n$ ou menos símbolos?
    \end{enumerate}
    %
    \begin{proof}[Solução]
        \begin{enumerate}
            \item Como há duas opções para cada símbolo, temos $2^n$ letras.
            \item Temos
            %
            \[
            \sum_{i=1}^n 2^i = \frac{2(2^n - 1)}{2 - 1} = 2(2^n - 1).
            \]
            %
        \end{enumerate}
    \end{proof}
    %
    \item Um homem possui $n$ chaves das quais, exatamente uma abre a fechadura. Ele experimenta as chaves uma de cada vez, escolhendo ao acaso em cada tentativa uma das chaves que não foram experimentadas. Determine a probabilidade de que ele escolha a chave correta na $r$-ésima tentativa?
    %
    \begin{proof}[Solução]
        A probabilidade é
        %
        \[
        \frac{n-1}{n}\cdot\frac{n-2}{n-1}\cdots\frac{n-(r-1)}{n-(r-2)}\cdot\frac{1}{n-(r-1)} = 1/n.
        \]
        %
    \end{proof}
    %
    \item Um ônibus parte com 6 pessoas e para em 10 pontos diferentes. Supondo que os passageiros têm igual probabilidade de saltar em qualquer parada, determine a probabilidade de que dois passageiros não desembarquem na mesma parada.
    %
    \begin{proof}[Solução]
        Temos
        %
        \[
        \Omega = \{ (x_1, x_2, \dots, x_6) : 1\leq x_i\leq 10, i = 1,2,\dots,6 \},
        \, \forall \mathcal{A} = \mathcal{P}(\Omega)
        \]
        %
        e $P:\mathcal{A}\to\mathbb{R}$ dada por $P(A) = |A|/|\Omega|$. Sendo $A$ o
        evento ``não desembarcar na mesma parada'', temos que
        %
        \[
        P(A) = |A|/|\Omega| = A_{10,6}/10^6.
        \]
        %
    \end{proof}
    %
    \item Suponha que temos $r$ caixas. Bolas são colocadas aleatoriamente nas caixas, uma de cada vez, até que alguma caixa contenha duas bolas pela primeira vez. Determine a probabilidade de que isto ocorra na $n$-ésima bola, com $r \geq n - 1$.
    %
    \begin{proof}[Solução]
        A probabilidade é $\dfrac{(n-1)A_{r, n-1}}{r^n}$.
    \end{proof}
    %
    \item Supondo que se distribui $n$ bolas em $n$ caixas.
    \begin{enumerate}[a)]
    \item Qual a probabilidade de que exatamente uma caixa esteja vazia?
    \item Dado que a caixa 1 está vazia, qual a probabilidade de que somente uma caixa esteja vazia?
    \item Dado que somente uma caixa está vazia, qual a probabilidade de que a caixa 1 esteja vazia.
    \end{enumerate}
    %
    \begin{proof}[Solução]
        \begin{enumerate}[a)]
            \item A probabilidade é $\binom{n}{2}n!/n^n$.
            \item Se a caixa 1 está vazia, as outras têm de estar preenchidas. Portanto,
            a probabilidade é $\binom{n}{2}(n-1)!/(n-1)^n$.
            \item A probabilidade de uma caixa $j$ qualquer estar vazia é $1/n$.
        \end{enumerate}
    \end{proof}
    %
    \item Se distribuímos aleatoriamente $n$ bolas em $r$ caixas, qual é a probabilidade de que a caixa 1 contenha $j$ bolas, com $0 \leq j \leq n$?
    %
    \begin{proof}[Solução]
        A probabilidade é $\binom{n}{j}(r-1)^{n-j}/r^n$.
    \end{proof}
    %
    \item Uma caixa contém $b$ bolas pretas e $r$ bolas vermelhas. Bolas são extraídas sem reposição, uma de cada vez. Determine a probabilidade de obter a primeira bola preta na $n$-ésima extração.
    %
    \begin{proof}[Solução]
        A probabilidade é
        %
        \[
        \frac{\binom{r}{n-1}}{\binom{r+b}{n-1}}\cdot\frac{b}{r+b-n+1}.
        \]
        %
    \end{proof}
    %
    \item Considere um baralho com 52 cartas. Uma mão de pôquer consiste de 5 cartas extraídas do baralho sem reposição e sem consideração da ordem. Considera-se que constituem sequências as mãos dos seguintes tipos: A, 2, 3,4,5; 2, 3, 4, 5, 6; ... ; 10, J, Q, K, A. Determine a probabilidade de ocorrência de cada uma das seguintes mãos de pôquer:
    \begin{enumerate}[a)]
    \item Royal flush ((10, J, Q, K, A) do mesmo naipe);
    \item Straight flush (cinco cartas do mesmo naipe em sequência);
    \item Four (valores da forma $(x, x, x, x, y)$ onde $x$ e $y$ são distintos);
    \item Full House (valores da forma $(x, x, x, y, y)$ onde $x$ e $y$ são distintos);
    \item Flush (cinco cartas do mesmo naipe);
    \item Straight (cinco cartas em sequência, sem consideração de naipes);
    \item Trinca (valores da forma $(x, x, x, y, z)$ onde $x$, $y$ e $z$ são distintos);
    \item Dois pares (valores da forma $(x, x, y, y, z)$ onde $x$, $y$ e $z$ são distintos);
    \item Um par (valores da forma $(w, w, x, y, z)$ onde $w$, $x$, $y$ e $z$ são distintos).
    \end{enumerate}
    %
    \begin{proof}[Solução]
        \begin{enumerate}[a)]
            \item $4/\binom{52}{5}$.
            \item $4\cdot 10/\binom{52}{5} = 4\cdot 10\cdot q$.
            \item $13\cdot 4\cdot 12\cdot q$.
            \item $13\cdot 4\cdot 12\cdot 6\cdot q$.
            \item $4\cdot\binom{13}{5}\cdot q$.
            \item $10\cdot 4^5\cdot q$.
            \item $13\cdot\binom{12}{2}\cdot 4^3\cdot q$.
            \item $\binom{13}{2}\cdot\binom{4}{2}\cdot\binom{4}{2}\cdot 44\cdot q$.
            \item $13\cdot\binom{4}{2}\cdot\binom{12}{3}\cdot 4^3\cdot q$.
        \end{enumerate}
    \end{proof}
    %
    \item Uma caixa contém dez bolas numeradas de 1 a 10. Seleciona-se uma amostra aleatória de 3 elementos. Determine a probabilidade de que as bolas 1 e 6 estejam entre as bolas selecionadas.
    %
    \begin{proof}[Solução]
        Dado que 1 e 6 estão selecionadas, há 8 possíveis bolas que completam o conjunto. Logo,
        a probabilidade desejada é $8/\binom{10}{3}$.
    \end{proof}
    %
    \item  Suponha que se extrai, sem reposição, uma amostra de tamanho $n$ de uma população de $r$ elementos. Obtenha a probabilidade de que $k$ objetos dados estejam incluídos na amostra.
    %
    \begin{proof}[Solução]
        Aqui, podemos imaginar que os $k$ elementos já foram incluídos: resta então escolher
        $n-k$ entre os $r-k$ vizinhos restantes. A probabilidade procurada é, portanto,
        $\binom{r-k}{n-k}/\binom{r}{n}$.
    \end{proof}
    %
    \item Qual a probabilidade de que 4 cartas extraídas de um baralho, 2 sejam pretas e 2 vermelhas?
    %
    \begin{proof}[Solução]
        A probabilidade é $\binom{26}{2}\binom{26}{2}/\binom{52}{4}$.
    \end{proof}
    %
    \item Se você possui 3 bilhetes de uma loteria para a qual se vendeu $n$ bilhetes e existem 5 prêmios, qual a probabilidade de você ganhar pelos menos um prêmio?
    %
    \begin{proof}[Solução]
        $1 - \frac{\binom{5}{0}\binom{n-5}{3}}{\binom{n}{3}}.$
    \end{proof}
    %
\end{enumerate}

\end{document}