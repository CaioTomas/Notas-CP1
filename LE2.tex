\documentclass[../Notas.tex]{subfiles}
\graphicspath{{\subfix{../images/}}}

\begin{document}

\subsection{Exercícios - combinatória}

\begin{enumerate}
    \item O código Morse consiste de uma sequência de pontos e traços em que repetições são permitidas.
    \begin{enumerate}[a)]
    \item Quantas letras podem ser codificadas usando exatamente $n$ símbolos?
    \item Qual é o número de letras que se pode codificar usando $n$ ou menos símbolos?
    \end{enumerate}
    \item Um homem possui $n$ chaves das quais, exatamente uma abre a fechadura. Ele experimenta as chaves uma de cada vez, escolhendo ao acaso em cada tentativa uma das chaves que não foram experimentadas. Determine a probabilidade de que ele escolha a chave correta na $r$-ésima tentativa?
    \item Um ônibus parte com 6 pessoas e para em 10 pontos diferentes. Supondo que os passageiros têm igual probabilidade de saltar em qualquer parada, determine a probabilidade de que dois passageiros não desembarquem na mesma parada.
    \item Suponha que temos $r$ caixas. Bolas são colocadas aleatoriamente nas caixas, uma de cada vez, até que alguma caixa contenha duas bolas pela primeira vez. Determine a probabilidade de que isto ocorra na $n$-ésima bola, com $r \geq n - 1$.
    \item Supondo que se distribui $n$ bolas em $n$ caixas.
    \begin{enumerate}[a)]
    \item Qual a probabilidade de que exatamente uma caixa esteja vazia?
    \item Dado que a caixa 1 está vazia, qual a probabilidade de que somente uma caixa esteja vazia?
    \item Dado que somente uma caixa está vazia, qual a probabilidade de que a caixa 1 esteja vazia.    
    \end{enumerate}
    \item Se distribuímos aleatoriamente $n$ bolas em $r$ caixas, qual é a probabilidade de que a caixa 1 contenha $j$ bolas, com $0 \leq j \leq n$?
    \item Uma caixa contém $b$ bolas pretas e $r$ bolas vermelhas. Bolas são extraídas sem reposição, uma de cada vez. Determine a probabilidade de obter a primeira bola preta na $n$-ésima extração.
    \item Considere um baralho com 52 cartas. Uma mão de pôquer consiste de 5 cartas extraídas do baralho sem reposição e sem consideração da ordem. Considera-se que constituem sequências as mãos dos seguintes tipos: A, 2, 3,4,5; 2, 3, 4, 5, 6; ... ; 10, J, Q, K, A. Determine a probabilidade de ocorrência de cada uma das seguintes mãos de pôquer:
    \begin{enumerate}[a)]
    \item Royal flush ((10, J, Q, K, A) do mesmo naipe);
    \item Straight flush (cinco cartas do mesmo naipe em sequência);
    \item Four (valores da forma $(x, x, x, x, y)$ onde $x$ e $y$ são distintos);
    \item Full House (valores da forma $(x, x, x, y, y)$ onde $x$ e $y$ são distintos);
    \item Flush (cinco cartas do mesmo naipe);
    \item Straight (cinco cartas em sequência, sem consideração de naipes);
    \item Trinca (valores da forma $(x, x, x, y, z)$ onde $x$, $y$ e $z$ são distintos);
    \item Dois pares (valores da forma $(x, x, y, y, z)$ onde $x$, $y$ e $z$ são distintos);
    \item Um par (valores da forma $(w, w, x, y, z)$ onde $w$, $x$, $y$ e $z$ são distintos).
    \end{enumerate}
    \item Uma caixa contém dez bolas numeradas de 1 a 10. Seleciona-se uma amostra aleatória de 3 elementos. Determine a probabilidade de que as bolas 1 e 6 estejam entre as bolas selecionadas.
    \item  Suponha que se extrai, sem reposição, uma amostra de tamanho $n$ de uma população de $r$ elementos. Obtenha a probabilidade de que $k$ objetos dados estejam incluídos na amostra.
    \item Qual a probabilidade de que 4 cartas extraídas de um baralho, 2 sejam pretas e 2 vermelhas?
    \item Se você possui 3 bilhetes de uma loteria para a qual se vendeu $n$ bilhetes e existem 5 prêmios, qual a probabilidade de você ganhar pelos menos um prêmio?
\end{enumerate}

\end{document}