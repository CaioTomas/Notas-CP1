\documentclass[../Notas.tex]{subfiles}
\graphicspath{{\subfix{../images/}}}

\begin{document}

\subsection{Exercícios - v.a.'s discretas}

\begin{enumerate}
    \item Qualquer ponto no intervalo $[0, 1)$ pode ser representado por meio de sua expansão decimal $0, x_1x_2\cdots$; suponha que se escolhe, aleatoriamente, um ponto do intervalo $[0, 1)$. Seja $X$ o primeiro dígito da expansão decimal que representa o ponto. Determine a função de probabilidade de $X$.
    %
    \begin{proof}[Solução]
        Como $X$ é uma v.a. uniforme, segue que
        %
        \[
        p_X(x) = \begin{cases}
        1/10, x\in\{0, 1, \dots, 9\} \\
        0, \text{c.c.}
        \end{cases}.
        \]
        %
    \end{proof}
    %
    \item \begin{enumerate}[a)]
        \item Se $X\sim B(n, p)$, determine a função de probabilidade de $Y = n - X$.
        \item Se $X\sim \text{Geom}(p)$, determine a função de probabilidade de $Y = X - 1$.
        \item Se $X$ tem uma função de probabilidade Binomial Negativa de parâmetros $r$ e $p$, (sendo $r$ inteiro e $0 < p < 1$), determine a função de probabilidade de $Y = X - r$.
    \end{enumerate}
    %
    \begin{proof}[Solução]
        %
        \begin{enumerate}[a)]
            \item Temos
            %
            \[
            p_X(k) = \begin{cases}
            \binom{n}{k}p^k(1-p)^{n-k}, k = 0,1,\dots,n \\
            0, \text{c.c}
            \end{cases},
            \]
            %
            logo
            %
            \[
            p_Y(k) = \begin{cases}
            \binom{n}{n-k}p^{n-k}(1-p)^{k}, k = 0, 1, \dots, n \\
            0, \text{c.c}
            \end{cases},
            \]
            %
            ou seja, $Y\sim B(n, 1-p)$.
            \item Temos
            %
            \[
            p_Y(k) = \begin{cases}
            (1-p)^kp, k = 0, 1, \dots \\
            0, \text{c.c.}
            \end{cases}.
            \]
            %
            \item Temos
            %
            \[
            p_Y(k) = \begin{cases}
            \binom{r+k-1}{r-1}p^r(1-p)^{k}, k = 0, 1, \dots \\
            0, \text{c.c.}
            \end{cases}.
            \]
            %
        \end{enumerate}
        %
    \end{proof}
    %
    \item Suponha que uma caixa contenha 6 bolas vermelhas e 4 pretas. Seleciona-se uma amostra aleatória de tamanho $n$. Seja $X$ o número de bolas vermelhas na amostra. Determine a função de probabilidade de $X$ para a amostragem:
    \begin{enumerate}[a)]
    \item sem reposição;
    \item com reposição.
    \end{enumerate}
    %
    \begin{proof}[Solução]
        \begin{enumerate}[a)]
            \item Temos $X\sim Hgeo(10,6,n)$.
            \item Temos $X\sim B(n, 6/10)$.
        \end{enumerate}
    \end{proof}
    %
    \item Seja N um número inteiro positivo e seja
    \begin{align*}
        p(x) = \begin{cases}
        c2^x, x\in\{1, 2, \dots, N\} \\
        0, \text{ c.c.}
        \end{cases}
    \end{align*}
    Determine o valor de $c$ para o qual $p$ é uma função de probabilidade.
    %
    \begin{proof}[Solução]
        Devemos ter $\sum_i p(x_i) = 1$, ou seja,
        %
        \[
        \sum_{i=1}^N c2^i = 1 \iff 2c(2^N - 1) = 1 \iff c = \frac{1}{2(2^N - 1)}.
        \]
        %
    \end{proof}
    %
    \item Suponha que $X$ tem uma distribuição geométrica com $p = 0,8$. Determine as probabilidades dos seguintes eventos:
    \begin{enumerate}[a)]
    \item $P(X > 3)$
    \item $P(4 \leq X \leq 7 \text{ ou } X > 9)$;
    \item $P(3 \leq X \leq 5 \text{ ou } 7 \leq X \leq 10)$;
    \end{enumerate}
    %
    \begin{proof}[Solução]
        \begin{enumerate}
            \item Temos
            %
            \[
            P(X>3) = 1 - P(X\leq 3)
                   = 1 - P(X=1) - P(X=2) - P(X=3)
                   = 0,008.
            \]
            %
            \item Usando a f.d. de $X$, temos
            %
            \[
            P(4\leq X\leq 7 \text{ ou } X>9) = P(X\leq 7) - P(X < 4) + 1 - P(X\leq 9)
                                             = 0,2^3 - 0,2^7 + 0,2^9.
            \]
            %
            \item Analogamente ao item b, temos que a probabilidade desejada é
            %
            \[
            P(X\leq 5) - P(X < 3) + P(X\leq 10) - P(X < 7)
            = 0,2^2 - 0,2^5 + 0,2^6 - 0,2^{10}.
            \]
            %
        \end{enumerate}
    \end{proof}
    %
    \item Suponha que $X$ tem uma distribuição uniforme sobre $0, 1, \dots , 99$. Determine:
    \begin{enumerate}[a)]
    \item $P(X \geq 25)$;
    \item $P(2,6 < X < 12,2)$;
    \item $P(8 < X \leq 10 \text{ ou } 30 < X \leq 32)$;
    \item $P(25 \leq X \leq 30)$
    \end{enumerate}
    %
    \begin{proof}[Solução]
        \begin{enumerate}
            \item Temos
            %
            \[
            P(X\geq 25) = 1 - P(X<25) = 1 - P(X\leq 24) = 1 - 25/100 = 3/4.
            \]
            %
            \item Temos
            %
            \[
            P(2,6<X<12,2) = P(3\leq X\leq 12) = P(X\leq 12) - P(X\leq 2) = 1/10.
            \]
            %
            \item Temos
            %
            \[
            P(8 < X \leq 10 \text{ ou } 30 < X \leq 32)
            = P(9\leq X\leq 10) + P(31\leq X\leq 32)
            = 1/25.
            \]
            %
            \item Temos
            %
            \[
            P(25\leq X\leq 30) = P(X\leq 30) - P(X\leq 24) = 3/50.
            \]
            %
        \end{enumerate}
    \end{proof}
    %
    \item Suponha que o número de chegadas de clientes em um posto de informações turísticas seja uma variável aleatória com distribuição Poisson com taxa de 2 pessoas por hora ($X\sim\text{Poisson}(2)$). Para uma hora qualquer, determine a probabilidade de ocorrer:
    \begin{enumerate}[a)]
    \item pelo menos uma chegada;
    \item mais de duas chegadas, dado que chegaram menos de 5 pessoas.
    \end{enumerate}
    %
    \begin{proof}[Solução]
        \begin{enumerate}[a)]
            \item A probabilidade desejada é dada por
            %
            \[
            P(X\geq 1) = 1 - P(X<1) = 1 - P(X\leq 0) = 1 - P(X=0) = 1 - e^{-2}.
            \]
            %
            \item A probabilidade desejada é
            %
            \[
            P(X>2|X<5) = \frac{P(2 < X < 5)}{P(X<5)} = \frac{P(3\leq X\leq 4)}{P(X\leq 4)}
            = \frac{2e^{-2}}{7e^{-2}} = \frac{2}{7}.
            \]
            %
        \end{enumerate}
    \end{proof}
    %
    \item Suponha que uma caixa contém 12 bolas numeradas de 1 a 12. Faz-se duas repetições independentes do experimento de selecionar aleatoriamente uma bola da caixa. Seja $X$ o maior entre os dois números observados. Determine a função de probabilidade de $X$.
    %
    \begin{proof}[Solução]
        Temos
        %
        \[
        p_X(k) = \begin{cases}
        \frac{2k-1}{144}, k = 1, 2, \dots, 12 \\
        0, \text{c.c.}
        \end{cases}.
        \]
        %
    \end{proof}
    %
    \item Considere a situação do Exercício 8 em que a seleção é feita sem reposição. 
    \begin{enumerate}[a)]
    \item Determine a função de probabilidade de $X$;
    \item Determine a função de distribuição de $X$.
    \end{enumerate}
    %
    \begin{proof}[Solução]
        \begin{enumerate}[a)]
            \item Temos
            %
            \[
            p(k) = \begin{cases}
            \frac{k-1}{\binom{12}{2}}, k=1,2\dots, 12 \\
            0, \text{c.c.}
            \end{cases}.
            \]
            %
            \item Temos
            %
            \[
            F_X(x) = \begin{cases}
            0, x < 2 \\
            \frac{\binom{[x]}{2}}{\binom{12}{2}}, 2\leq x < 12 \\
            1, x\geq 12.
            \end{cases}
            \]
            %
        \end{enumerate}
    \end{proof}
    %
    \item Suponha que uma caixa contenha $r$ bolas numeradas de 1 a $r$. Seleciona-se sem reposição uma amostra aleatória de tamanho $n$. Seja $Y$ o maior número observado na amostra e $Z$ o menor. Determine:
    \begin{enumerate}[a)]
    \item $P(Y \leq y), \forall y \in \mathbb{R}$;
    \item $P(Z \geq z), \forall y \in \mathbb{R}$.
    \end{enumerate}
    %
    \begin{proof}[Solução]
        \begin{enumerate}[a)]
            \item Temos
            %
            \[
            P(Y\leq y) = \begin{cases}
            0, y<n \\
            \frac{\binom{[y]}{n}}{\binom{r}{n}}, n\leq y < r \\
            1, y\geq r.
            \end{cases}
            \]
            %
            \item Temos
            %
            \[
            P(Z\geq z) = \begin{cases}
            1, z<1 \\
            \frac{\binom{r+1-z}{n}}{\binom{r}{n}}, z = 1, 2, \dots, r-n+1 \\
            \frac{\binom{r-[z]}{n}}{\binom{r}{n}}, i < z < i+1, i = 1, 2, \dots, r-n \\
            0, z > r-n+1
            \end{cases}.
            \]
            %
        \end{enumerate}
    \end{proof}
    %
    \item Considere $X$ uma variável aleatória assumindo valores em $\{0, \pm 1, \pm 2\}$. Suponha que $P(X = -2) = P(X = -1)$ e $P(X = 1) = P (X = 2)$ com a informação que $P(X > 0) = P(X < 0) = P(X = 0)$. Encontre a função de probabilidade e a função de distribuição de $X$.
    %
    \begin{proof}[Solução]
        Sejam $P(X=-2) = x = P(X=-1)$, $P(X=2) = y = P(X=1)$ e $P(X=0) = z$. De
        $P(X>0) = P(X=0) = P(X<0)$, temos $2x = 2y = z$. Daí, como $2x + 2y + z = 1$, temos
        $x = y = 1/6$ e $z=1/3$. Portanto,
        %
        \[
        p_X(k) = \begin{cases}
        1/6, k = \pm1, \pm2 \\
        1/3, k=0 \\
        0, \text{c.c.}
        \end{cases}
        \]
        %
        e
        %
        \[
        F_X(k) = \begin{cases}
        0, x<-2 \\
        1/6, -2\leq x < -1 \\
        1/3, -1\leq x < 0 \\
        2/3, 0\leq x < 1 \\
        5/6, 1\leq x < 2 \\
        1, x\geq 2
        \end{cases}.
        \]
        %
    \end{proof}
    %
    \item Seja $X$ uma v.a. com função de distribuição dada por:
    \begin{align*}
        F_X(x) = \begin{cases}
            0, x < 0 \\
            x/2, 0\leq x < 1 \\
            2/3, 1\leq x < 2 \\
            11/12, 2\leq x < 3 \\
            1, x\geq 3
        \end{cases}.
    \end{align*}
    Determine
    \begin{enumerate}[a)]
        \item $P(X < 3)$; 
        \item $P(X = 1)$; 
        \item $P (X > 1/2 )$; 
        \item $P(2 < X \leq 4)$; 
        \item $P(2 \leq X \leq 4)$.
    \end{enumerate}
    %
    \begin{proof}[Solução]
        \begin{enumerate}[a)]
            \item Temos $P(X<3) = F(3^{-}) = 11/12$.
            \item Temos $P(X=1) = F(1) - F(1^-) = 1/6$.
            \item Temos $P(X>1/2) = 1 - F(1/2) = 3/4$.
            \item Temos $P(2 < X\leq 4) = F(4) - F(2) = 1/12$.
            \item Temos $P(2\leq X\leq 4) = F(4) - F(2^-) = 1/3$.
        \end{enumerate}
    \end{proof}
    %
\end{enumerate}

\end{document}