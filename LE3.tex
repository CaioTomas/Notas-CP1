\documentclass[../Notas.tex]{subfiles}
\graphicspath{{\subfix{../images/}}}

\begin{document}

\subsection{Exercícios - v.a.'s discretas}

\begin{enumerate}
    \item Qualquer ponto no intervalo $[0, 1)$ pode ser representado por meio de sua expansão decimal $0, x_1x_2\cdots$; suponha que se escolhe, aleatoriamente, um ponto do intervalo $[0, 1)$. Seja $X$ o primeiro dígito da expansão decimal que representa o ponto. Determine a função de probabilidade de $X$.
    \item \begin{enumerate}[a)]
        \item Se $X\sim B(n, p)$, determine a função de probabilidade de $Y = n - X$.
        \item Se $X\sim \text{Geom}(p)$, determine a função de probabilidade de $Y = X - 1$.
        \item Se $X$ tem uma função de probabilidade Binomial Negativa de parâmetros $r$ e $p$, (sendo $r$ inteiro e $0 < p < 1$), determine a função de probabilidade de $Y = X - r$.
    \end{enumerate}
    \item Suponha que uma caixa contenha 6 bolas vermelhas e 4 pretas. Seleciona-se uma amostra aleatória de tamanho $n$. Seja $X$ o número de bolas vermelhas na amostra. Determine a função de probabilidade de $X$ para a amostragem:
    \begin{enumerate}[a)]
    \item sem reposição;
    \item com reposição.
    \end{enumerate}
    \item Seja N um número inteiro positivo e seja
    \begin{align*}
        p(x) = \begin{cases}
        c2^x, x\in\{1, 2, \dots, N\} \\
        0, \text{ c.c.}
        \end{cases}
    \end{align*}
    Determine o valor de $c$ para o qual $p$ é uma função de probabilidade.
    \item Suponha que $X$ tem uma distribuição geométrica com $p = 0,8$. Determine as probabilidades dos seguintes eventos:
    \begin{enumerate}[a)]
    \item $P(X > 3)$
    \item $P(4 \leq X \leq 7 \text{ou} X > 9)$;
    \item $P(3 \leq X \leq 5 \text{ou} 7 \leq X \leq 10)$;
    \end{enumerate}
    \item Suponha que $X$ tem uma distribuição uniforme sobre $0, 1, \dots , 99$. Determine:
    \begin{enumerate}[a)]
    \item $P(X \geq 25)$;
    \item $P(2,6 < X < 12,2)$;
    \item $P(8 < X \leq 10 ou 30 < X \leq 32)$;
    \item $P(25 \leq X \leq 30)$
    \end{enumerate}
    \item Suponha que o número de chegadas de clientes em um posto de informações turísticas seja uma variável aleatória com distribuição Poisson com taxa de 2 pessoas por hora ($X\sim\text{Poisson}(2)$). Para uma hora qualquer, determine a probabilidade de ocorrer:
    \begin{enumerate}[a)]
    \item pelo menos uma chegada;
    \item mais de duas chegadas, dado que chegaram menos de 5 pessoas.
    \end{enumerate}
    \item Suponha que uma caixa contém 12 bolas numeradas de 1 a 12. Faz-se duas repetições independentes do experimento de selecionar aleatoriamente uma bola da caixa. Seja $X$ o maior entre os dois números observados. Determine a função de probabilidade de $X$.
    \item Considere a situação do Exercício 8 em que a seleção é feita sem reposição. 
    \begin{enumerate}[a)]
    \item Determine a função de probabilidade de $X$;
    \item Determine a função de distribuição de $X$.
    \end{enumerate}
    \item Suponha que uma caixa contenha $r$ bolas numeradas de 1 a $r$. Seleciona-se sem reposição uma amostra aleatória de tamanho $n$. Seja $Y$ o maior número observado na amostra e $Z$ o menor. Determine:
    \begin{enumerate}[a)]
    \item $P(Y \leq y), \forall y \in \mathbb{R}$;
    \item $P(Z \geq z), \forall y \in \mathbb{R}$.
    \end{enumerate}
    \item Considere $X$ uma variável aleatória assumindo valores em $\{0, \pm 1, \pm 2\}$. Suponha que $P(X = -2) = P(X = -1)$ e $P(X = 1) = P (X = 2)$ com a informação que $P(X > 0) = P(X < 0) = P(X = 0)$. Encontre a função de probabilidade e a função de distribuição de $X$.
    \item Seja $X$ uma v.a. com função de distribuição dada por:
    \begin{align*}
        F_X(x) = \begin{cases}
            0, x < 0 \\
            x/2, 0\leq x < 1 \\
            2/3, 1\leq x < 2 \\
            11/12, 2\leq x < 3 \\
            1, x\geq 3
        \end{cases}.
    \end{align*}
    Determine
    \begin{enumerate}[a)]
        \item $P(X < 3)$; 
        \item $P(X = 1)$; 
        \item $P (X > 2 )$; 
        \item $P(2 < X \leq 4)$; 
        \item $P(2 \leq X \leq 4)$.
    \end{enumerate}
    
    
    
\end{enumerate}

\end{document}