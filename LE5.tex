\documentclass[../Notas.tex]{subfiles}
\graphicspath{{\subfix{../images/}}}

\begin{document}

\subsection{Exercícios - esperança e funções de v.a.'s discretas}

\begin{enumerate}
    \item Seja $X$ uma v.a. com função de distribuição dada por:
    \begin{align*}
        F_X(x) = \begin{cases}
            0, x < -2 \\
            1/8, -2\leq x < 1 \\
            5/8, 1\leq x < 2 \\
            7/8, 2\leq x < 4 \\
            1, x\geq 4
        \end{cases}.
    \end{align*}
    Determine:
    \begin{enumerate}[a)]
        \item a função de probabilidade de $X$;
        \item $EX$.
    \end{enumerate}
    %
    \begin{proof}[Solução]
        \begin{enumerate}[a)]
            \item Temos
            %
            \[
            p_X(x) = \begin{cases}
            1/8, x=-2,4 \\
            1/2, x=1 \\
            1/4, x=2
            \end{cases}
            \]
            %
            \item Temos
            %
            \[
            EX = -2/8 + 4/8 + 1/2 + 2/4 = 5/4.
            \]
            %
        \end{enumerate}   
    \end{proof}
    %
    \item Seja $X$ uma v.a. com função de probabilidade dada por:
    \begin{align*}
        p_X(x) = \begin{cases}
        \frac{1}{2|x|(|x|+1)}, x = \pm 1, \pm 2, \dots \\
        0, \text{ c.c.}
        \end{cases}.
    \end{align*}
    \begin{enumerate}[a)]
        \item Calcule $\displaystyle{ \sum_{x\in\mathbb{Z}} |x|p_X(x) }$.
        \item Mostre que $EX$ não existe.
    \end{enumerate}
    %
    \begin{proof}[Solução]
        \begin{enumerate}[a)]
            \item Temos
            %
            \[
            \sum_{x\in\mathbb{Z}}|x|p_X(x) = \frac{1}{2}\sum_{x\in\mathbb{Z}} \frac{1}{|x|+1} = \infty.
            \]
            %
            \item Como a série acima diverge, não existe $EX$.
        \end{enumerate}   
    \end{proof}
    %
    \item Seja $N$ um número inteiro positivo e seja $p$ a função definida por
    \begin{align*}
        p = \begin{cases}
            \frac{2x}{N(N+1)}, x\in\{1, 2, \dots N\} \\
            0, \text{ c.c.}
        \end{cases}.
    \end{align*}
    \begin{enumerate}[a)]
        \item Mostre que $p$ é uma função de probabilidade.
        \item Seja $X$ v.a. com f.p. $p$; calcule $EX$.
    \end{enumerate}
    %
    \begin{proof}[Solução]
        \begin{enumerate}
            \item Temos $p(x)\geq 0$ para todo $x\in \mathbb{R}$, o conjunto $\{1, 2, \dots, N\}$
            é finito e
            %
            \[
            \sum_{x=1}^N \frac{2x}{N(N+1)} = 1.
            \]
            %
            \item Temos
            %
            \[
            \sum_x |x|p(x) = \sum_{x=1}^N \frac{2x^2}{N(N+1)} = \frac{2N+1}{3}.
            \]
            %
        \end{enumerate}
    \end{proof}
    %
    \item Suponha que $X$ se distribui uniformemente em $\{1, \dots , N\}$. Determine $EX$ e $E[X^2]$.
    %
    \begin{proof}[Solução]
        Temos
        %
        \[
        EX = \sum_{x=1}^N \frac{x}{N} = \frac{N+1}{2}
        \]
        %
        e
        %
        \[
        EX^2 = \sum_{x=1}^N \frac{x^2}{N} = \frac{(N+1)(2N+1)}{6}.
        \]
        %
    \end{proof}
    %
    \item Suponha que $X$ tem distribuição binomial de parâmetros $n = 4$ e $p$. Obtenha $E[\sin(\pi X/2)]$.
    %
    \begin{proof}[Solução]
        Temos
        %
        \[
        E[\sin(\pi X/2)] = \sum_{x=0}^3 \sin(\pi x/2)\binom{4}{x}p^x(1-p)^{4-x}
                         = 4p(1-p)(1-2p).
        \]
        %
    \end{proof}
    %
    \item Suponha que $X$ tem distribuição de Poisson de parâmetro $\lambda$. Determine $\displaystyle{E[\frac{1}{1+X}]}$.
    %
    \begin{proof}[Solução]
        Temos
        %
        \[
        E[1/(1+X)] = \sum_{x=0}^{\infty} \frac{1}{1+x}e^{-\lambda}\frac{\lambda^x}{x!}
                   = \frac{e^{-\lambda}}{\lambda}(e^{\lambda} - 1)
                   = \frac{1 - e^{-\lambda}}{\lambda}.
        \]
        %
    \end{proof}
    %
    \item Seja $X$ uma variável aleatória com distribuição geométrica de parâmetro $p$ e seja $M > 0$ um número inteiro positivo. Determine a esperança das seguintes v.a.’s:
    \begin{enumerate}[a)]
    \item $Z = \min(X, M)$
    \item $W = \max(X, M)$.
    \end{enumerate}
    %
    \begin{proof}[Solução]
        \begin{enumerate}[a)]
            \item A esperança é
            %
            \[
            \sum_{x=1}^{\infty} |\min(x,M)|p(1-p)^{x-1} 
            = \sum_{x=1}^{\infty} \min(x,M)p(1-p)^{x-1}
            = \frac{1 - (1-p)^M}{p}
            \]
            %
            após algumas simplificações.
            \item A esperança é
            %
            \[
            \sum_{x=1}^{\infty} |\max(x,M)|p(1-p)^{x-1} 
            = \sum_{x=1}^{\infty} \max(x,M)p(1-p)^{x-1}
            = M + \frac{(1-p)^M}{p}
            \]
            %
            após algumas simplificações.
        \end{enumerate}
    \end{proof}
    %
    \item Em ensaios de Bernoulli independentes, com probabilidade $p$ de sucesso, sejam $X$ o número de ensaios até a ocorrência do $r$-ésimo sucesso e $Y$ o número de fracassos anteriores ao $r$-ésimo sucesso. Determine $EX$ e $EY$.
    %
    \begin{proof}[Solução]
        Note que $X = \sum_{i=1}^r T_i$, sendo $T_i\sim\text{Geo}(p)$, $i=1,2,\dots,r$.
        Daí, $EX = r/p$ e, além disso, $Y = X-r$, ou seja, $Y = \sum_{i=1}^r T_i - r$.
        Logo, $EY = r/p - r = r(1-p)/p$.
    \end{proof}
    %
    \item Seja $(X,Y)$ um vetor aleatório com função de probabilidade conjunta dada por:
    \begin{align*}
        p_{X,Y}(x,y)\begin{cases}
            p, x = \pm 1, y = 0 \\
            1-2p, x = 0, y = 1 \\
            0, \text{ c.c.}
        \end{cases},
    \end{align*}
    onde $0 < p < 1/2$. Verifique que $E[XY] = EXEY$, mas $X$ e $Y$ não são independentes.
    %
    \begin{proof}[Solução]
        Temos
        %
        \[
        E(XY) = \sum_{x=-1}^1\sum_{y=0}^1 xyp_{X,Y}(x,y) = 0.
        \]
        %
        Ademais, $EXEY = 0(1-2p) = 0 = E(XY)$. Ora, mas 
        $P(X=0, Y=1) = 1-2p \neq (1-2p)^2 = P(X=0)P(Y=1)$, logo $X$ e $Y$ não são independentes.
    \end{proof}
    %
\end{enumerate}

\end{document}