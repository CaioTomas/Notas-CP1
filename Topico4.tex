\documentclass[../Notas.tex]{subfiles}
\graphicspath{{\subfix{../images/}}}

\begin{document}

%\section{Tópico 4}

\section{Esperança de variáveis aleatórias discretas}
Começamos com uma motivação para depois introduzirmos a definição de esperança: dada uma amostra $\{ a_1, a_2, \dots, a_n \}\subset\mathbb{R}$, a média amostral é dada por $\displaystyle{ \frac{a_1 + a_2 + \cdots + a_n}{n} }$. Podemos representar esses dados numa tabela:
\begin{table}[H]
    \centering
    \begin{tabular}{c|c}
        Valores distintos na amostra & Frequência absoluta \\
        \hline
        $x_1$ & $N_1$ \\
        $x_2$ & $N_2$ \\
        \vdots & \vdots \\
        $x_m$ & $N_m$
    \end{tabular}
%    \caption{Caption}
%    \label{tab:my_label}
\end{table}
Note que $N_i$ é o número de vezes que $x_i$ aparece na amostra, $i= 1,2,\dots, m$. Além disso, $N_1 + N_2 + \cdots + N_m = n$ e $x_1N_1 + x_2N_2 + \cdots + x_mN_m = a_1 + a_2 + \cdots + a_n$. Com isso, podemos escrever a média dos $a_i$ como $\displaystyle{ \frac{x_1N_1 + x_2N_2 + \cdots + x_mN_m}{N_1 + N_2 + \cdots + N_m} }$ ou, ainda, $\displaystyle{ x_1\frac{N_1}{n} + \cdots + x_m\frac{N_m}{n} }$, em que $N_i/n$ é a frequência relativa de $x_i$, $i=1,2,\dots,m$. Nesse caso, a f.p. $p:\mathbb{R}\to\mathbb{R}$ é dada por
\begin{align*}
    p(x) = \begin{cases}
    N_i/n, x=x_i \text{ para algum } i=1,2,\dots,m \\
    0, \text{ c.c.}
    \end{cases}.
\end{align*}
Agora, suponhamos que $x_1, x_2, \dots, x_m$ sejam valores possíveis de uma v.a. $X$ e que $a_1, a_2, \dots, a_n$ são valores (independentes) observados de $X$. Então, de acordo com a interpretação da probabilidade como frequência relativa (Lei Forte dos Grandes Números), temos que para $n$ grande a frequência relativa se aproxima da probabilidade real, i.e.,
\begin{align*}
    \lim_{n\to\infty}\frac{N_i}{n} = p(x_i) = P(X=x_i).
\end{align*}
Assim, o \textbf{valor esperado} de $X$, representado por $EX$ ou $E[X]$, é $\displaystyle{ EX = \sum_{i=1}^{m}x_i p(x_i) }$, sendo $p$ a f.p. de $X$.

\begin{definition}[Esperança]
Seja $X$ uma v.a. discreta em $(\Omega, \mathcal{A}, P)$ com valores possíveis $\{ x_1, x_2, \dots \}$. Se $\displaystyle{ \sum_{i} |x_i|p(x_i) < \infty }$, dizemos que $X$ tem \textbf{esperança finita} e definimos sua esperança ou \textbf{esperança matemática} ou \textbf{valor esperado} ou \textbf{média} como
\begin{align*}
    EX = \sum_{i=1}^{\infty}x_ip(x_i) = \sum_x xp(x).
\end{align*}
Note, em particular, que se o conjunto dos valores possíveis de $X$ é finito, então $X$ tem esperança finita.
\end{definition}

\begin{example}
Seja $X\sim B(n,p)$. Temos
\begin{align*}
    p_X(k) = \begin{cases}
    \binom{n}{k}p^k(1-p)^{n-k}, k = 0,1,\dots,n \\
    0, \text{ c.c.}
    \end{cases}.
\end{align*}
Como $X$ tem uma quantidade finita de valores possíveis, então $X$ tem esperança finita dada por
\begin{align*}
    EX = \sum_x xp(x) = \sum_{k=0}^{n} kP(X=k) = \sum_{k=1}^{n} k\binom{n}{k}p^k(1-p)^{n-k} &= \sum_{k=1}^{n} \frac{n!}{(k-1)!(n-k)!}p^k(1-p)^{n-k} \\
    &= np\sum_{k=1}^{n} \binom{n-1}{k-1}p^{k-1}(1-p)^{n-1-(k-1)} \\
    &= np.
\end{align*}
Em particular, se $n=1$ então $EX = p$.
\end{example}

\begin{example}
Seja $X\sim\text{Poisson}(\lambda)$. Temos
\begin{align*}
    p_X(k) = \begin{cases}
    e^{-\lambda}\frac{\lambda^k}{k!}, k = 0,1,\dots \\
    0, \text{ c.c.}
    \end{cases}.
\end{align*}
Note que como $X$ tem infinitos valores possíveis, não sabemos a priori se $EX < \infty$. Temos
\begin{align*}
    \sum_x |x|p_X(x) = \sum_{k=0}^{\infty} |k|e^{-\lambda}\frac{\lambda^k}{k!} = \lambda e^{-\lambda}\sum_{k=1}^{\infty}\frac{\lambda^{k-1}}{(k-1)!} = \lambda < \infty.
\end{align*}
Logo, $X$ tem esperança finita e, como seus valores possíveis são não negativos, segue que $EX=\lambda$.
\end{example}

\begin{example}
Seja $X\sim\text{Geom}(p)$. Temos
\begin{align*}
    p_X(k) = \begin{cases}
    p(1-p)^{k-1}, k = 1,2,\dots \\
    0, \text{ c.c.}
    \end{cases}, 0 < p < 1.
\end{align*}
Note que
\begin{align*}
    \sum_x |x|p_X(x) = \sum_{k=1}^{\infty} |k|p(1-p)^{k-1} = p\sum_{k=1}^{\infty} k(1-p)^{k-1} &= -p\sum_{k=1}^{\infty} \dfrac{d}{dp}[(1-p)^k] \\
    &= -p\dfrac{d}{dp}\left[ \sum_{k=1}^{\infty} (1-p)^k \right] \\
    &= -p(-1/p^2) \\
    &= 1/p,
\end{align*}
em que a série converge pelo teste da integral. Logo, $X$ tem esperança finita e, como seus valores possíveis são não negativos, temos $EX=1/p$.
\end{example}

\begin{example}

\end{example}



\subsection{Propriedades}



\subsection{Momentos}



\end{document}