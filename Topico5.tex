\documentclass[../Notas.tex]{subfiles}
\graphicspath{{\subfix{../images/}}}

\begin{document}

%\section{Tópico 5}

\section{Variáveis aleatórias contínuas}
Vimos situações em que as v.a.'s representavam o número de ``objetos'' ou ``coisas''. Entretanto, há muitas situações (tanto teóricas quanto práticas) em que a v.a. natural a se considerar é ``contínua'' num certo sentido, e.g. o tempo que decorre até a recuperação completa de um paciente com determinada doença.

\begin{definition}[V.a. contínua]
Uma v.a. $X$ em $(\Omega, \mathcal{A}, P)$ é dita \textbf{contínua} se $P(X=x) = 0, \forall x\in\mathbb{R}$.
\end{definition}

Recordando das propriedades da f.d. de uma v.a., temos o seguinte fato.

\begin{proposition}
$X$ é v.a. contínua se, e só se, $F_X$ é contínua.
\end{proposition}

\begin{proof}
Dado $x\in\mathbb{R}$,
\begin{align*}
    0 = P(X=x) = F_X(x) - F_X(x^-) \iff F_X(x^+) = F_X(x) = F_X(x^-).
\end{align*}
\end{proof}

No caso de v.a.'s contínuas, podemos trocar os sinais $<$ e $\leq$ à vontade nos cálculos de probabilidades.

\begin{example}
Considere o experimento de escolher um ponto ao acaso no círculo de centro na origem e raio $R>0$ (jogar um dardo). Vimos que, nesse caso, um espaço de probabilidade adequado é $(\Omega, \mathcal{A}, P)$ com
\begin{align*}
    \Omega = \{ (x,y)\in\mathbb{R}^2 : x^2+y^2\leq R^2 \}, \mathcal{A} = \mathcal{B}^2(\Omega)
\end{align*}
e $P:\mathcal{A}\to\mathbb{R}$ tal que
\begin{align*}
    P(A) = \iint_A \frac{1}{\pi R^2}dxdy, \forall A\in\mathcal{A}.
\end{align*}
Podemos definir a v.a. $X:\Omega\to\mathbb{R}$ por $X(\omega) = X((x,y)) = \sqrt{x^2+y^2}$. Note que, dado $z\in\mathbb{R}$, temos $\{X = z\} = \emptyset$ se $z < 0$ ou $z>R$ e $\{ X=z \} = \{ (x,y)\in\Omega : x^2+y^2 = z^2 \}$ se $0\leq z \leq R$. Logo, $P(X=z) = 0, \forall z\in\mathbb{R}$, ou seja, $X$ é v.a. contínua. Por outro lado, $\{X\leq z\} = \emptyset$ se $z<0$, $\{X\leq z\} = \{ (x,y)\in\Omega : x^2+y^2 \leq z^2 \}$ se $0\leq z < R$ e $\{X\leq z\} = \Omega$ se $z\geq R$. Logo,
\begin{align*}
    F_X(z) = \begin{cases}
    0, z < 0 \\
    z^2/R^2, 0\leq z < R \\
    1, z\geq R
    \end{cases}.
\end{align*}
Em particular, se $0\leq a < b\leq R$, então
\begin{align*}
    P(a < X < b) = F_X(b) - F_X(a) = \frac{b^2 - a^2}{R^2} > 0.
\end{align*}
\begin{center}
    \textbf{GRÁFICO P.104}
\end{center}
\end{example}

\begin{definition}[Função de densidade]
Uma função $f:\mathbb{R}\to\mathbb{R}$ tal que
\begin{enumerate}[(i)]
    \item $f(x)\geq 0, \forall x\in\mathbb{R}$ 
    \item $\displaystyle{ \int_{\mathbb{R}} f(x)dx  = 1}$
\end{enumerate}
é dita \textbf{função de densidade de probabilidade} ou apenas \textbf{densidade}.
\end{definition}

\begin{definition}[Função de distribuição]
Uma função $F:\mathbb{R}\to\mathbb{R}$ tal que
\begin{align*}
    F(x) = \int_{-\infty}^x f(t) dt, \forall x\in\mathbb{R}
\end{align*}
para alguma densidade $f$ é dita \textbf{função de distribuição (absolutamente) contínua}. Dizemos ainda que $f$ é a densidade de $F$.
\end{definition}

\begin{remark}
É possível, mas complicado, construir exemplos de funções $F$ que sejam contínuas mas não tenham densidade. As que têm densidade são chamadas de \textbf{absolutamente} contínuas. Aqui não faremos distinção, pois os casos não absolutamente contínuos são raros; sempre que nos referirmos a uma f.d., estará implícito que ela é absolutamente contínua.

Outro ponto importante é que dada $F$, a densidade $f$ não é única, já que $F$ pode não ser derivável. Contudo, os pontos onde $F$ não é derivável (ou onde $f$ não é contínua) formam um conjunto enumerável, de maneira que a integral não se altera. Em geral é comum tomar $f$ como
\begin{align*}
    f(x) = \begin{cases}
    F'(x), \forall x\in\mathbb{R} : \exists F'(x) \\
    0, \text{ c.c.}
    \end{cases}
\end{align*}
\end{remark}

\begin{definition}
Uma v.a. $X$ definida em $(\Omega, \mathcal{A}, P)$ é \textbf{(absolutamente) contínua} se sua f.d. $F_X$ é \textbf{(absolutamente) contínua}, i.e., se $F_X(x) = \displaystyle{ \int_{-\infty}^x f(t)dt, \forall x\in\mathbb{R} }$ para alguma função de densidade $f:=f_X$, chamada \textbf{densidade} de $X$.
\end{definition}

\begin{remark}
Como no caso discreto, podemos nos referir tanto a $F_X$ quanto a $f_X$ quando dizemos ``distribuição'', devido à relação biunívoca entre ambas as funções. Além disso, se $X$ é v.a. contínua com densidade $f_X$, então dados $a,b\in\mathbb{R}$ quaisquer com $a\leq b$, temos
\begin{align*}
    P(a < X < b) = F_X(b) - F_X(a) = \int_a^b f_X(x) dx,
\end{align*}
ou seja, $P(a < X < b)$ é a área da região $A$.
\begin{center}
    \textbf{GRÁFICO P.105}
\end{center}
\end{remark}

\begin{example}
No exemplo do dardo acima, vimos que
\begin{align*}
    F_X(x) = \begin{cases}
    0, x < 0 \\
    x^2/R^2, 0\leq x < R \\
    1, x\geq R
    \end{cases}.
\end{align*}
Logo, a densidade $f$ de $X$ é
\begin{align*}
    f_X(x) = \begin{cases}
    0, x < 0 \\
    2x/R^2, 0\leq x < R \\
    0, x\geq R
    \end{cases}.
\end{align*}
Note que não existe $F'_X(R)$, pois $F_{X_-}'(R) = 2/R \neq 0 = F'_{X_+}(R)$, ou seja, $f$ não é contínua em $R$. Assim, sem perda de generalidade, tomamos $f(R) = 0$, de modo que
\begin{align*}
    f_X(x) = \begin{cases}
    2x/R^2, 0\leq x < R \\
    0, \text{ c.c.}
    \end{cases}
\end{align*}
e ainda vale $\displaystyle{ F_X(x) = \int_{-\infty}^x f_X(t) dt, \forall x\in\mathbb{R}. }$
\end{example}

\begin{remark}
É importante notar que há v.a.'s que não são nem contínuas nem discretas, chamadas \textbf{mistas}. Por exemplo, a v.a. $X$ com f.d. $F_X$ dada pelo gráfico abaixo não é contínua, pois $F_X$ é descontínua em $a$; entretanto, $X$ tampouco é discreta, pois $F_X$ não é do tipo escada.
\begin{center}
    \textbf{GRÁFICO P.106}
\end{center}
\end{remark}

\begin{definition}
Seja $X$ uma v.a.; dizemos que
\begin{enumerate}[(i)]
    \item $X$ é simétrica em torno de 0 se $P(X\geq x) = P(X\leq -x), \forall x\in\mathbb{R}$, i.e., $X$ e $-X$ têm a mesma distribuição;
    \item $X$ é simétrica em torno de $\mu$ se existe $\mu\in\mathbb{R}$ tal que $P(X\geq \mu + x) = P(X\leq \mu - x), \forall x\in\mathbb{R}$.
\end{enumerate}
\end{definition}

\begin{theorem}
Seja $X$ v.a. contínua com densidade $f$. Então $X$ é simétrica em torno de 0 se, e só se, $f$ é par; nesse caso, $f$ é \textbf{densidade simétrica}. De modo geral, $X$ é simétrica em torno de $\mu\in\mathbb{R}$ se, e só se, $f(\mu + x) = f(\mu - x)$. Nesse caso, $f$ é \textbf{densidade simétrica em torno de} $\mu$.
\end{theorem}

\begin{proof}
Provamos em torno de 0. Se $f$ é par, então
\begin{align*}
    P(X\geq x) = \int_x^{\infty} f(t)dt = \int_{-\infty}^{-x} f(-y)dy = \int_{-\infty}^{-x} f(y)dy = P(X\leq -x),
\end{align*}
logo $X$ é simétrica. Reciprocamente, se $P(X\geq x) = P(X\geq -x)$, defina
\begin{align*}
    g(x) = \frac{f(x) + f(-x)}{2}.
\end{align*}
Note que $g$ é simétrica, logo
\begin{align*}
    \int_{-\infty}^x g(y) dy &= \frac{1}{2}\int_{-\infty}^x f(y) dy + \frac{1}{2}\int_{-\infty}^x f(-y) dy \\
    &= \frac{1}{2}\int_{-\infty}^x f(y) dy + \frac{1}{2}\int_{-x}^{\infty} f(y) dy \\
    &= \frac{1}{2}P(X\leq x) + \frac{1}{2}P(-X\geq -x) \\
    &= P(X\leq x),
\end{align*}
ou seja, $X$ tem densidade $g$ simétrica. A demonstração do caso geral é análoga, substituindo $x$ por $x+\mu$.
\end{proof}

\begin{center}
    \textbf{GRÁFICO P.106}
\end{center}

\begin{remark}
É possível mostrar que o resultado acima vale também para v.a.'s discretas; além disso, note que se $X$ é simétrica em torno da origem então $F(0) = 1/2$ e, se $X$ é simétrica em torno de $\mu$, então $F(\mu) = 1/2$. De maneira geral, se $X$ é simétrica em torno da origem então
\begin{align*}
    F(-x) = \int_{-\infty}^{-x}f(y) dy = \int_x^{\infty} f(-y)dy = \int_x^{\infty} f(y) dy = \int_{-\infty}^{\infty} f(y) dy - \int_{-\infty}^x f(y) dy,
\end{align*}
ou seja, $F(-x) = 1 - F_X(x), \forall x\in\mathbb{R}$.
\end{remark}

\subsection{Exemplos clássicos de distribuições contínuas}
\begin{example}[Uniforme contínua, $X\sim U(a,b)$]
Dizemos que $X$ é uma v.a. contínua com distribuição uniforme no intervalo $(a,b)$ se tem densidade dada por
\begin{align*}
    f(x) = \begin{cases}
    \frac{1}{b-a}, a < x < b \\
    0, \text{ c.c.}
    \end{cases}.
\end{align*}
Note que $f(x)\geq 0, \forall x\in\mathbb{R}$ e $\displaystyle{ \int_{\mathbb{R}} f(x) dx = 1 }$. A f.d. $F$ associada a $f$ é calculada como segue:
\begin{align*}
    F(x) &= \int_{-\infty}^x f(t) dt = 0, x < a \\
    F(x) &= \int_{-\infty}^x f(t) dt = \int_a^x \frac{1}{b-a} dt = \frac{x-a}{b-a}, a\leq x < b \\
    F(x) &= \int_{-\infty}^x f(t) dt = \int_a^b \frac{1}{b-a} dt = 1, x \geq b, 
\end{align*}
ou seja,
\begin{align*}
    F(x) = \begin{cases}
    0, x<a \\
    \frac{x-a}{b-a}, a\leq x < b \\
    1, x\geq b
    \end{cases}.
\end{align*}
Note também que $f(x) = F'(x), \forall x\in\mathbb{R}\setminus\{a,b\}$.
\end{example}

\begin{example}[Exponencial, $X\sim\text{Exp}(\lambda)$]
Dizemos que $X$ é uma v.a. contínua com distribuição exponencial de parâmetro $\lambda > 0$ se tem densidade dada por
\begin{align*}
    f(x) = \begin{cases}
    \lambda e^{-\lambda x}, x > 0 \\
    0, x \leq 0
    \end{cases}
\end{align*}
ou, equivalentemente, f.d. dada por
\begin{align*}
    F(x) = \begin{cases}
    1 - e^{-\lambda x}, x > 0 \\
    0, x\leq 0
    \end{cases}.
\end{align*}
Note que, de fato, $f(x) \geq 0, \forall x\in\mathbb{R}$ e que
\begin{align*}
    \int_{\mathbb{R}} f(t) dt = \int_0^{+\infty} \lambda e^{-\lambda t} dt = 1.
\end{align*}
Ademais, $f(x) = F'(x), \forall x\in\mathbb{R}\setminus\{0\}$.
\end{example}

\begin{remark}
A distribuição exponencial é utilizada muitas vezes quando a v.a. em questão é um tempo de espera, e.g. o tempo até que um componente eletrônico apresente falhas. Além disso, se $X\sim\text{Exp}(\lambda)$, então $X$ tem perda de memória, i.e., $P(X > a+b) = P(X>a)P(X>b), a,b\geq 0$ ou, equivalentemente, $P(X>a+b| X>a) = P(X>b), a,b\geq 0$.
\begin{proof}
De fato, se $X\sim\text{Exp}(\lambda)$, então
\begin{align*}
    P(X>a+b|X>a) &= \frac{P(X>a+b, X>a)}{P(X>a)} \\
    &= \frac{P(X>a+b)}{P(X>a)} \\
    &= \frac{ e^{-\lambda (a+b)} }{ e^{-\lambda a} } \\
    &= e^{-\lambda b} \\
    &= P(X>b), \forall a,b\geq 0.
\end{align*}
\end{proof}
Na verdade, a v.a. exponencial é a única v.a. contínua não negativa com perda de memória.
\end{remark}

\begin{theorem}
$X$ é v.a. contínua com perda de memória se, e só se, $X\sim\text{Exp}(\lambda)$ para algum $\lambda > 0$ ou $P(X>0) = 0$.
\end{theorem}

\begin{proof}
($\Leftarrow$) Já vimos o caso que $X$ tem distribuição exponencial; se $P(X>0) = 0$, então $P(X > a+b) = 0 = P(X>a)P(X>b), \forall a,b\geq 0$.

($\Rightarrow$) Se $X$ tem perda de memória e $P(X > 0) \neq 0$, então tomando $a = 0 = b$ segue que
\begin{align*}
    P(X>0) = [P(X>0)]^2 \implies P(X > 0) = 1, 
\end{align*}
ou seja, $X$ é v.a. positiva. Seja $F$ a f.d. de $X$ e defina $G(x) = 1 - F(x)$. Temos $G$ não crescente, contínua à direita, $G(0) = 1, G(+\infty) = 0$ e $G(a+b) = G(a)G(b), \forall a,b>0$. Daí, se $c\in\mathbb{R}^*_+$ e $m,n\in\mathbb{R}$, temos
\begin{align*}
    G(c) = G(c - c/m)G(c/m) = G(c - 2c/m)[G(c/m)]^2 = \cdots = G(0)[G(c/m)]^m = [G(c/m)]^m,
\end{align*}
donde segue que
\begin{align*}
    G(nc) = [G(c)]^n.
\end{align*}
Além disso, temos $0 < G(1) < 1$. De fato, se $G(1) = 1$ então teríamos
\begin{align*}
    G(n) = [G(1)]^n \implies G(+\infty) = 1,
\end{align*}
absurdo. Se $G(1) = 0$, então teríamos
\begin{align*}
    G(1/m) = 0 \implies G(0) = G(0^+) = \lim_{m\to +\infty} G(1/m) = \lim_{m\to +\infty} [G(1)]^{1/m} = 0,
\end{align*}
absurdo. Logo, como $0 < G(1) < 1$, podemos tomar $G(1) = e^{-\lambda}$, para algum $\lambda > 0$. Tomando $c=1$, segue que $G(1/m) = e^{-\lambda /m}, m\in\mathbb{N}$ e, fazendo $c = 1/m$, temos $G(n/m) = [G(1/m)]^n = e^{-\lambda n/m}, \forall n,m\in\mathbb{N}$. Logo, $G(y) = e^{-\lambda y}, \forall y\in\mathbb{Q}$. Da continuidade à direita, temos
\begin{align*}
    G(x) = \lim_{y\to x^+, y\in\mathbb{Q}} G(y) = e^{-\lambda x}, \forall x\in\mathbb{R}^*_+,
\end{align*}
pois $\mathbb{Q}$ é denso em $\mathbb{R}$. Daí, segue que $F(x) = 1 - e^{-\lambda x}, \forall x>0$, ou seja, $X\sim\text{Exp}(\lambda)$.
\end{proof}

\begin{remark}
Além de tempo de falha, variáveis exponenciais são úteis para estudar o tempo de decaimento de uma partícula radioativa e, também, no estudo de processos de Poisson e cadeias de Markov. Ademais, pensando na variável exponencial como o tempo de falha, a propriedade de perda de memória diz que dado que não houve falha até o tempo $a$, a probabilidade de que não haja falha nas próximas $b$ unidades de tempo é igual à probabilidade incondicional de que não haja falha nas primeiras $b$ unidades de tempo. Isso implica que o desgaste de uma peça de equipamento não aumenta nem diminui a probabilidade de falha em um dado intervalo de tempo.
\end{remark}

\begin{example}[Normal/gaussiana]
Dizemos que $X$ é v.a. com distribuição normal \textbf{padrão}, $X\sim N(0,1)$, se tem densidade dada por
\begin{align*}
    f(x) = \frac{1}{\sqrt{2\pi}}e^{-x^2/2}, x\in\mathbb{R}.
\end{align*}
\begin{proof}
Vamos mostrar que $f$ de fato é densidade. Primeiro, $f(x) > 0, \forall x\in\mathbb{R}$. Ademais, vamos verificar que $\displaystyle{ \int_{\mathbb{R}} f(x) dx = 1 }$. Seja $g(x) = e^{-x^2/2}, x\in\mathbb{R}$. Note que $g$ é par, contínua e não negativa. Além disso, se $x\geq 1$ então $0 < g(x) < e^{-x/2}$ e, então,
\begin{align*}
    \int_{1}^{\infty} e^{-x^2/2} dx \leq \int_{1}^{\infty} e^{-x/2} dx = \lim_{a\to +\infty} \int_{1}^{a} e^{-x/2} dx = \lim_{a\to +\infty} -2e^{-x/2}\Big|_{1}^a = 2e^{-1/2}\in\mathbb{R},
\end{align*}
logo $\displaystyle{ \int_{1}^{\infty} e^{-x^2/2} dx \in\mathbb{R}}$. Como $g$ é par, temos $\displaystyle{ \int_{1}^{\infty} e^{-x^2/2} dx = \int_{-\infty}^{-1} e^{-x^2/2} dx \in\mathbb{R}}$ e, como $g$ é contínua em $\mathbb{R}$, temos também $\displaystyle{ \int_{-1}^{1} e^{-x^2/2} dx \in\mathbb{R}}$. Logo, $\displaystyle{ \int_{-\infty}^{\infty} e^{-x^2/2} dx \in\mathbb{R}}$, digamos $c$. Segue então que
\begin{align*}
    c^2 = \int_{-\infty}^{\infty} e^{-x^2/2} dx\int_{-\infty}^{\infty} e^{-y^2/2} dy = \iint_{\mathbb{R}^2} e^{-\frac{x^2+y^2}{2}} dxdy = \int_{0}^{\infty}\int_{0}^{2\pi} re^{-r^2/2} drd\theta = 2\pi \int_{0}^{\infty} re^{-r^2/2} dr = 2\pi,
\end{align*}
ou seja, $c = \sqrt{2\pi}$, pois $c>0$. Logo, $\displaystyle{ \int_{\mathbb{R}} f(x) dx = \frac{1}{\sqrt{2\pi}} \int_{\mathbb{R}} g(x) dx = 1.}$ Note que $f$ é simétrica em torno de 0 e, além disso, a f.d. $F$ associada a $f$ é dada por
\begin{align*}
    F(x) = \int_{-\infty}^x \frac{1}{\sqrt{2\pi}} e^{-t^2/2} dt, x\in\mathbb{R}.
\end{align*}
Verifica-se que $F$ não tem uma forma fechada, mas podemos aproximar seus valores numericamente. Como mencionado antes, temos $F(0) = 1/2 = 1 - F(0)$ e $F(x) = 1 - F(x), \forall x\in\mathbb{R}$. É comum também denotar $f$ por $\varphi$ e $F$ por $\Phi$ no caso da densidade e da f.d. de uma v.a. normal padrão, respectivamente.
\end{proof}
\begin{remark}
Tabelas de valores para a distribuição normal geralmente fornecem as probabilidades do tipo $P(0 < X < a)$. De fato, basta apenas esta probabilidade, pois $P(X\leq a) = 1/2 + P(0 < X < a), a>0$ e analogamente para os demais casos.
\end{remark}
De maneira geral, dizemos que $X$ é v.a. contínua com distribuição \textbf{normal de parâmetros} $\mu$ \textbf{e} $\sigma^2$, $X\sim N(\mu, \sigma^2)$ (veremos o que esses parâmetros significam mais à frente), se tem densidade dada por
\begin{align*}
    f(x) = \frac{1}{\sqrt{2\pi}\sigma}e^{-\frac{(x-\mu)^2}{2\sigma^2}}, x\in\mathbb{R},
\end{align*}
com $\mu,\sigma\in\mathbb{R}$ e $\sigma>0$. A verificação de que $f$ de fato é densidade é análoga ao que fizemos acima, bastando apenas efetuar a mudança de variável $y = \displaystyle{\frac{x-\mu}{\sigma}}$. Note também que $f$ é simétrica em torno de $\mu$.
\begin{center}
    \textbf{GRÁFICO P.110}
\end{center}
As variáveis aleatórios com distribuição normal ocorrem frequentemente em aplicações práticas. A Lei de Maxwell da Física afirma que, sob condições adequadas, as componentes da velocidade de uma molécula de gás estarão aleatoriamente distribuídas seguindo uma distribuição normal $N(0,\sigma^2)$, onde $\sigma^2$ depende de certas quantidades físicas. Entretanto, na maioria das aplicações, as v.a.'s de interesse têm distribuições que é \textbf{aproximadamente} normal. Por exemplo, erros instrumentais em experimentos físicos e variabilidade biológica (e.g., altura e massa) foram verificados, empiricamente, como possuindo distribuições aproximadamente normais. Veremos esse tipo de comportamente mais adiante.
\end{example}

\begin{example}[Gama, $X\sim\Gamma(\alpha, \lambda)$]
Dizemos que
\end{example}



\subsection{Funções de variáveis aleatórias contínuas}

\subsection{Funções de distribuição inversas}

\section{Vetores aleatórios contínuos}

\subsection{Definições gerais}

\subsection{Distribuições marginais e independência}

\subsection{Função de distribuição $n$-dimensional}

\subsection{Funções de vetores aleatórios}

\subsection{Densidades condicionais}

\subsubsection{Regra de Bayes}

\section{Esperança de variáveis aleatórias contínuas}

\subsection{Definição}

\subsection{Esperança de função de variável aleatória contínua}

\subsection{Momentos de uma variável aleatória contínua}

\subsection{Esperança condicional}




\end{document}